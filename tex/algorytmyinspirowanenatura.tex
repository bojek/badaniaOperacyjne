Algorytmy inspirowane naturą w ostatnich latach zyskały dużą popularność ze względu na fakt,
że wiele rzeczywistych problemów optymalizacyjnych stały się bardziej dynamiczne oraz złożone.
Rozmiar i złożoność problemów w dzisiejszych czasach wymaga rozwoju wielu metod,
których efektywność jest mierzona na podstawie zdolności do znalezienia zadowalających wyników w rozsądnym czasie,
a nie tylko do zagwarantowania optymalnego rozwiązania. Celem naukowców stała się więc obserwacja natury,
próba zrozumienia efektywnych algorytmów wykorzystywanych przez inne organizmy żywe.
Przeniesienie tej wiedzy i dopasowanie rozwiązań na grunt informatyki - specjaliści oczekują uzyskania jak najlepszych rezultatów.