Problem komiwojażera (bądź jak ktoś woli TSP - travelling salesman problem) jest jednym z najstarszych - a przynajmniej
jednym z najbardziej popularnych problemów optymalizacyjnych. Zapewne wiele osób z tym problemem już się spotkało i wie,
że jest związany z teorią grafów. Ściślej rzecz ujmując sprowadza się do znalezienia cyklu Hamiltona w grafie pełnym
(graf musi być pełny, aby istniało rozwiązanie).\\

\noindent Kilka słów wyjaśnień:

\begin{description}
\item [Graf pełny] - jest to graf w którym dla każdej pary wierzchołków istnieje krawędź je łącząca.
             W przypadku grafu pełnego o n wierzchołkach wiadomo, iż posiada dokładnie n(n-1)/2 krawędzi.
\item [Cykl Hamiltona] - jest to taki cykl, który zawiera każdy wierzchołek danego grafu dokładnie 1 raz (prócz pierwszego
                 wierzchołka, ponieważ w nim rozpoczyna się i kończy cykl).
\end{description}
Dodatkowo wiadomo, krawędziom grafu można nadawać różne wagi.\\

Wracając do problemu: dane jest n miast oraz odległość/koszt podróży/czas podróży między nimi. Nasz strudzony
komiwojażer musi odwiedzić wszystkie te miasta w celu dostarczenia przesyłek (cokolwiek stanowią) jednak z zachowaniem
zminimalizowanych kosztów. W zależności od problemu celem może być znalezienie najkrótszej/najtańszej/najszybszej trasy.
Warunek jest taki, że każdą miejscowość ma odwiedzić dokładnie 1 raz po czym ma wrócić do miejscowości w której "wojaże"
rozpoczął.\\

\noindent W odniesieniu do teorii grafów:
\begin{itemize}
\item wierzchołek grafu reprezentuje jedno miasto;
\item  krawędź reprezentuje drogę łączącą miasta;
\item  jako, że istnieją pewne kryteria (odległość/koszt/czas), krawędzie grafu mają przyporządkowane odpowiednie wagi.
\end{itemize}