\subsubsection{Co to jest karaluch?}
%\begin{table}[!htp]
%\begin{tabular}{p{3cm}p{9cm}}
%{Karaluch:}&{karaczan wschodni}\\
%{Gromada:}&{owady}\\
%{Podgromada:}&{uskrzydlone}\\
%{Rząd:}&{karaczany}\\
%\end{tabular}
%\end{table}
Niewielki owad żyjący w siedliskach człowieka. Żyje w kuchniach, magazynach, wyjadając produkty żywnościowe. Długość ciała dochodzi do 3 cm. Ubarwienie czerwone do czarno-brązowego.
Ciało owadów pokryte jest cienką, woskową powłoką, nie przepuszczającą wody. Czułki są zazwyczaj tej samej długości co ciało i stanowią główny narząd zmysłu. Karaczany mają narządy gębowe typu gryzącego. Na odwłoku występują często gruczoły wydzielające odstraszającą woń i ciecz. Pierwsza para skrzydeł jest skórzasta i tworzy pokrywę, druga para jest błoniasta.
Wykazują cechy stadne. Rozpoznają się wzajemnie dzięki specyficznemu wydzielanemu zapachowi. Podążają chaotycznie w miejsca gdzie są już inne osobniki. Niektóre gatunki prowadzą nocny tryb życia. Najchętniej gromadzą się w szparach i szczelinach oraz w miejscach wilgotnych. Mogą przenosić choroby, statystycznie rzecz biorąc, na każdym karaczanie przebywa nawet 80 chorobotwórczych bakterii.
\subsubsection{Ciekawostki} 
Karaluchy są wszystkożerne - są jednymi z niewielu zwierząt zdolnymi do strawienia keratyny - białka włosa.
W niektórych krajach ma nie tylko negatywny wizerunek w Rosji np. uznawany jest za dowód dostatku w domu, bo tam gdzie marnuje się jedzenie, tam są i karaluchy.
Niewymagający i niewybredny owad słynie z odporności, np. bez jedzenia może się obejść nawet 40 dni. Są one odporne na wiele trucizn i potrafią bez uszczerbku znieść promieniowanie radioaktywne w dawkach nawet 100 krotnie większych niż dawka szkodliwa dla człowieka. Znane są przypadki karaluchów żyjących mimo braku głowy
Zespół naukowców z Nottingham University odkrył w mózgach karaluchów i szarańczy aż 9 różnych substancji o silnym działaniu przeciwbakteryjnym, a siedem z nich zabijało 90 procent „superbakterii” , nie szkodząc przy tym ludzkim komórkom.

Z racji swojej odrazy do karaluchów ludzie nazywali znienawidzone owady tak jak nie lubiane nacje, np. jeden z krajowych gatunków nazywany jest w Polsce prusakiem (uznana nazwa naukowa!) lub popularnie szwabem.



