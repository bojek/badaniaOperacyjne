Obecnie jednym z podstawowych elementów zarządzania wszelkiego typu projektami jest proces harmonogramowania. Pod
"wszelkiego typu" możemy rozumieć niemalże każde przedśiewzięcie - inżyniera, sektor IT, inwestycje, przemysł transportowy
, budowlany itp. Za terminem harmonogramownia stoi nic innego jak planowanie najbardziej optymalnej sekwencji operacji
oraz zminimalizownie czasu wykonania tychże sekwencji. Określamy skończone ramy czasowe wykonania danych zadań oraz
znajdujemy możliwe równoległe scieżki wykonywania operacji. Oczywistą korzyścią harmonogramowania jest najlepsze
wykorzystaniu czasu oraz możliwośc dokładnego obliczenia terminu wykonania całego lub części przedsięwzięcia.
W ostatnich latach powstało wiele pokaźnej literatury poruszającej problem, zrodziło się wiele definicji, modeli oraz teorii. 